\section{Literature Review}

This section reviews relevant literature and resources foundational to the development of a Multi-Agent System (MAS) for managing autonomous vehicles 
at road intersections. 

These studies encompass Deep Reinforcement Learning (DRL) frameworks, applications in traffic systems, and recent advancements in multi-agent cooperation, robustness, and scalability.


\textbf{Foundational DRL Frameworks:}

\textit{rlagents: Implementations of Reinforcement Learning Algorithms}\cite{rl-agents} is a robust library offering implementations of DRL algorithms such as DQN, PPO, and A2C. 
The modular framework supports easy customization and integration into traffic management systems, offering a robust foundation for training agents in navigation tasks.
This library serves as the foundation for much of our work.

\textbf{Attention Mechanisms in Dense Traffic:}

Leurent and Mercat\cite{leurent2019socialattention} introduced the concept of social attention, which allows agents to prioritize essential interactions 
in dense traffic scenarios.

In our work, this concept has been incorporated into the agent's policy, demonstrating improved scalability and safety. 
This aligns closely with the project's second stage, focused on perturbation testing.

\textbf{Simulation Environments for Traffic Scenarios:}

\textit{Highway Multi-Agent Environment}\cite{highwaymultiagentenv} provides a simulation platform for multi-agent systems, supporting sparse 
and dense traffic configurations. 
It enables realistic testing and evaluation of trained policies, addressing the project's goals of scalability and robustness.

\textbf{Related Studies in DRL and Traffic Management:}

Wei et al.\cite{wei2019intellilight} developed IntelliLight, a DRL-based traffic signal control system, demonstrating enhanced traffic throughput 
and reduced delays. 

Yang et al.\cite{yang2018meanfield} proposed Mean Field MARL to address scalability challenges in environments with numerous agents. 

Chu et al.\cite{chu2019multiagent} applied MARL to intersection management, showcasing collaborative decision-making for optimizing intersection 
traffic flow.

Kiran et al.\cite{kiran2021survey} provided a survey on DRL in autonomous driving, highlighting challenges like environmental 
perturbations and model transferability. 

Chen et al.\cite{chen2021collisionavoidance} focused on collision avoidance using MARL in mixed traffic scenarios. 

Wang et al.\cite{wang2020multiintersection} examined cooperative MARL for multi-intersection traffic signal control, demonstrating the benefits of decentralized learning. 

Finally, Baker et al.\cite{baker2020emergent} explored emergent behaviors in MAS, emphasizing the importance of adaptive and cooperative policies. 

Thananjeyan et al.\cite{thananjeyan2021} developed a DRL approach integrating safety constraints, ensuring reliable decision-making in 
critical tasks.


